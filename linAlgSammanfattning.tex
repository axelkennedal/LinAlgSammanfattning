\documentclass[11pt]{article}

%svenska
\usepackage[swedish]{babel}
\usepackage[utf8]{inputenc}

\author{Axel Kennedal} % Skriv ditt namn här om du bidragit :)!
\title{Sammanfattning i Linjär Algebra}

%textutseende
\renewcommand{\familydefault}{\sfdefault}

%layout
\usepackage[dvipsnames]{xcolor}

%specialkommandon
\newenvironment{definition}[1]{\subsection{\textcolor{Cerulean}{Definition: #1}}}{}

%extramatte
\usepackage{amsmath}
\usepackage{mathtools}

\begin{document}
\maketitle
\tableofcontents
\newpage

\section{Baser \& Basbyten}
\definition{Bas}
En bas är grunden för ett vektorrum och utgörs av ett antal \emph{basvektorer}, vilka är linjärt oberoende och spänner upp vektorrummet.

Ex: en bas i R2
$$
B = \{\vec{v_{1}}, \vec{v_{2}} \}
$$

\definition{Standardbas}
En bas i RN med n st basvektorer vars element är bara nollor förutom en etta i den kolumnen m för den m-te basvektorn i basen.

\begin{equation}
B = \left \{
\begin{bmatrix}
1 \\
0 \\
\vdots \\
0 \\
\end{bmatrix}
,
\begin{bmatrix}
0 \\
1 \\
\vdots \\
0 \\
\end{bmatrix}
, \cdots
,
\begin{bmatrix}
0 \\
0 \\
\vdots \\
1 \\
\end{bmatrix}
\right \}
\end{equation}
\end{document}
