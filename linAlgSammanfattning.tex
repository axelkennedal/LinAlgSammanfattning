\documentclass[11pt]{article}

%svenska
\usepackage[swedish]{babel}
\usepackage[utf8]{inputenc}

\author{Axel Kennedal} % Skriv ditt namn här om du bidragit :)!
\title{Sammanfattning i Linjär Algebra}

%textutseende
\renewcommand{\familydefault}{\sfdefault}

%layout
\usepackage[dvipsnames]{xcolor}

%specialkommandon
\newenvironment{begrepp}[1]{\subsection{\textcolor{Cerulean}{Begrepp: #1}}}{}
\newcommand{\comment}[1]{\begin{center}\textcolor{Gray}{#1} \end{center}}

%extramatte
\usepackage{amsmath}
\usepackage{mathtools}

\begin{document}
\maketitle
\tableofcontents
\newpage

\section{Baser \& Basbyten}
\begrepp{Bas}
En bas är grunden för ett vektorrum och utgörs av ett antal \emph{basvektorer}, vilka är linjärt oberoende och spänner upp vektorrummet. Om dessa 2 krav uppfylls är \emph{dimensionen} för vektorrummet = antal basvektorer.
\begin{equation}
B = \{\vec{v_{1}}, \vec{v_{2}} \}
\end{equation}
\comment{En bas i R2}
\begrepp{Standardbas}
En bas i RN med n st basvektorer vars element är bara nollor förutom en etta i den kolumnen m för den m-te basvektorn i basen.
Är per definition \emph{ortonomal}.

\begin{equation}
B = \left \{
\begin{bmatrix}
1 \\
0 \\
\vdots \\
0 \\
\end{bmatrix}
,
\begin{bmatrix}
0 \\
1 \\
\vdots \\
0 \\
\end{bmatrix}
, \ldots
,
\begin{bmatrix}
0 \\
0 \\
\vdots \\
1 \\
\end{bmatrix}
\right \}
\end{equation}
\begrepp{Basbyte}
En vektor kan beskrivas i ett vektorrum med hjälp av \emph{koordinater}, denna koordinatrepresentation skiljer sig mellan olika vektorrum som har olika baser. Ibland vill man byta mellan baser 
då det kan ge enklare beräkningar.

\begrepp{Koordinater \& Koordinatvektor}
Om $S = \{\vec{v_{1}}, \ldots, \vec{v_{n}}\}$ är en bas för ett vektorrum och $\vec{v} = c_{1}\vec{v_{1}} + \cdots + c_{n}\vec{v_{n}}$ så är $c_{1},\ldots,c_{n}$ koordinater och $\vec{c} = (c_{1}, \ldots, c_{n}) = (\vec{v})_{S}$ koordinatvektorn för $\vec{v}$ i basen $S$ 

\end{document}
