\documentclass[10pt]{article}

%svenska
\usepackage[swedish]{babel}
\usepackage[utf8]{inputenc}

\author{Axel Kennedal \\ kennedal@kth.se \\\\
Marcus Östling \\} % Skriv ditt namn här om du bidragit :)!
\title{Sammanfattning i Linjär Algebra}

%textutseende
\renewcommand{\familydefault}{\sfdefault}

%layout
\usepackage[dvipsnames]{xcolor}
\usepackage[margin = 2cm]{geometry}
\setlength{\columnsep}{1cm}
%specialkommandon
\newcommand{\indelning} [1] {\subsection{#1}}
\newcommand{\begrepp} [1] {\subsubsection{\textcolor{Cerulean}{Begrepp: #1}}}
\newcommand{\metod} [1] {\subsubsection{\uppercase{\small{Beräkningsmetod: #1}}}}
\newcommand{\comment} [1] {\begin{center}\textcolor{Gray}{#1} \end{center}}

%extramatte
\usepackage{amsmath}
\usepackage{mathtools}

\begin{document}
\maketitle
\tableofcontents
\newpage

\newgeometry{margin = 1cm}
\twocolumn

\section{Matriser}
\indelning{Vanliga beräkningar}
\begrepp{Matris}
En uppställning av tal i rader och kolumner, storleken anges som $rader \times kolumner$ eller $n \times m$.

\metod{Matrisaddition}
Addera alla element var för sig.
\begin{displaymath}
\begin{bmatrix}
a \\
b \\
\end{bmatrix}
+
\begin{bmatrix}
c \\
d \\
\end{bmatrix}
=
\begin{bmatrix}
a + c \\
b + d \\
\end{bmatrix}
\end{displaymath}

\metod{Multiplikation med skalär}
Multiplicera varje element i matrisen med skalären.
\begin{displaymath}
s *
\begin{bmatrix}
a & b \\
c & d \\
\end{bmatrix}
=
\begin{bmatrix}
sa & sb \\
sc & sd \\
\end{bmatrix}
\end{displaymath}
\comment{$s\in R$}

\metod{Matrismultiplikation}
Endast definierad för två matriser A $m \times n$ och B $n \times  m$ vilket ger en matris C $m \times m$.
Multiplicera varje rad i A med B's kolumner.
\begin{displaymath}
\begin{bmatrix}
c & e \\
d & f \\
\end{bmatrix}
\begin{bmatrix}
a \\
b \\
\end{bmatrix}
=
\begin{bmatrix}
ca + eb \\
da + fb \\
\end{bmatrix}
\end{displaymath}
\comment{Multiplikation av en $2 \times 2$ och en $2 \times 1$ matris}

\begrepp{Identitetsmatrisen}
Multiplikation av en matris A med identitetsmatrisen ger tillbaka A oförändrad.
\begin{displaymath}
I_{3} =
\begin{bmatrix}
1 & 0 & 0 \\
0 & 1 & 0 \\
0 & 0 & 1 \\
\end{bmatrix}
\end{displaymath}
\comment{Identitetsmatrisen har ettor längs diagonalen och nollor på alla andra platser}

\begrepp{Symmetrisk matris}
Innebär att $A^T = A$.
Om A är symmetrisk och inverterbar är $A^{-1}$ symmetrisk.

\indelning{Invers}
\begrepp{Matrisinvers}
Krav för att en matris A ska vara inverterbar:
\begin{itemize}
\item $\exists B: AB = I$.
\item A är kvadratisk
\item $\det(A) \neq 0$
\item A's reducerade form är I
\end{itemize}
\section{Baser \& Basbyten}
\begrepp{Bas}
En bas är grunden för ett vektorrum och utgörs av ett antal \emph{basvektorer}, vilka är linjärt oberoende och spänner upp vektorrummet. Om dessa 2 krav uppfylls är \emph{dimensionen} för vektorrummet = antal basvektorer.
\begin{displaymath}
B = \{\vec{v_{1}}, \vec{v_{2}} \}
\end{displaymath}
\comment{En bas i R2}
\begrepp{Standardbas}
En bas i RN med n st basvektorer vars element är bara nollor förutom en etta i den kolumnen m för den m-te basvektorn i basen.
Är per definition \emph{ortonomal}.

\begin{displaymath}
B = \left \{
\begin{bmatrix}
1 \\
0 \\
\vdots \\
0 \\
\end{bmatrix}
,
\begin{bmatrix}
0 \\
1 \\
\vdots \\
0 \\
\end{bmatrix}
, \ldots
,
\begin{bmatrix}
0 \\
0 \\
\vdots \\
1 \\
\end{bmatrix}
\right \}
\end{displaymath}
\begrepp{Basbyte}
En vektor kan beskrivas i ett vektorrum med hjälp av \emph{koordinater}, denna koordinatrepresentation skiljer sig mellan olika vektorrum som har olika baser. Ibland vill man byta mellan baser 
då det kan ge enklare beräkningar.

\begrepp{Koordinater \& Koordinatvektor}
Om $S = \{\vec{v_{1}}, \ldots, \vec{v_{n}}\}$ är en bas för ett vektorrum och $\vec{v} = c_{1}\vec{v_{1}} + \cdots + c_{n}\vec{v_{n}}$ så är $c_{1},\ldots,c_{n}$ koordinater och $\vec{c} = (c_{1}, \ldots, c_{n}) = {(\vec{v})}_{S}$ koordinatvektorn för $\vec{v}$ i basen $S$. 

\end{document}
